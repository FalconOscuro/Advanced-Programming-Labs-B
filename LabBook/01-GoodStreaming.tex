\section{The Good Streaming Operator}
    \subsection*{Question}
    The code provided takes an integer and then a floating point number from the user and then displays the two numbers in the console window. \\
        
    This code creates an integer variable called \mintinline{cpp}{i} and a float variable called \mintinline{cpp}{f}. 
    It then takes the user input from the console \mintinline{cpp}{cin} by storing two numbers from the user into the two variables. 
    The code then outputs to the console the values of the two variables. \\
        
    Run this code and enter into the console window an integer followed by a space and then followed by a floating point number before pressing the return key (\textit{e.g. 23  4.567}). 
    The program should output the values of the two numbers.\\

    \begin{itemize}
        \item Add a third variable that is a string type. You will need to add \mintinline{CPP}{#include <string>}.

        \item Take a string value from the user after you take the float value (\textit{the user can enter something like 23  4.586  Hello into the console window}).

        \item Then output the string value to the console after you output the float value.
    \end{itemize}
            
    \subsection*{Test Data}
        Not required.
        
    \subsection*{Sample Output}
        \noindent\textit{
            Enter an int and a float separated by a ' '(space): 34 98.6\\
            Enter a string: asd\\
            i=34, f=98.6\\
            s=asd\\
        }

    \subsection*{Reflection}
        The program works as expected, but was unaware multiple values could be passed
        into and extracted from the input stream at once.
        
    \subsection*{Solution}
        \begin{listing}[H]
            \inputminted{cpp}{../Tasks/01-Good-Streaming/Good-Streaming.cpp}%CPP file path here
            \caption{Good-Streaming.cpp}
        \end{listing}
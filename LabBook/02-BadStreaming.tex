\documentclass[Lab-B.tex]{subfiles}

\begin{document}
    \section{The Bad Streaming Operator}
        \subsection*{Question:}
            The streaming operator \mintinline{CPP}{>>} can be used on most fundamental types and standard objects like string. However, 
            it can lead to unexpected issues when storing values into an array.\\

            Test the following code, compiled for both debug and release targets and 
            utilizing the debugger to view the state of varaible \mintinline{cpp}{c} in memory. \\
    
            \begin{minted}[autogobble]{cpp}
                #include <iostream>
                using namespace std;

                int main(int argn, char* argv[])
                {
                    char c[5];
                    cin >> c;
                    cout << "c=" << c << endl;
                }
            \end{minted}

            The \mintinline{cpp}{>>} streaming operator is great for storing values into different types, 
            but it should be avoided to store values into an array.\\

            Instead replace \mintinline{cpp}{cin >> c} with \mintinline{cpp}{cin.get(c, 5)}.\\
            This code will store up to 4 values followed by the end of array character. 
            Run your code as a normal Debug build and with the Debugger and inspect and document the values and see the output. 
            The \mintinline{cpp}{get()} function is usually safer for storing multiple characters in an array that the \mintinline{cpp}{>>} operator.
            
        \subsection*{Solution:}
            \inputminted{cpp}{../02-Bad-Streaming/Bad-Streaming.cpp}%CPP file path here

        \subsection*{Test Data:}
            Not Required.
        
        \subsection*{Sample Output:}
            \begin{center}
                \begin{tabular}{c c}
                    \hline
                    \multicolumn{2}{c}{\textbf{unsafe}} \\
                    \textbf{input} & \textbf{data stored} \\
                    \hline
                    asdfghj & 'a', 's', 'd', 'f', 'g' \\
                    \\
                    \hline
                    \multicolumn{2}{c}{\textbf{safe}} \\
                    \textbf{input} & \textbf{data stored} \\
                    \hline
                    asdfghj & 'a', 's', 'd', 'f', '\textbackslash000' \\
                    
                \end{tabular}
            \end{center}

        \subsection*{Reflection:}
            Writing beyond the bounds of assigned memory is known as stack smashing,
            this causes data loss and undefined behaviour. Modern operating systems
            have safeguards in place as this could be used as a method of attack,
            on my os even in debug mode it terminates the process due to this.    

\end{document}